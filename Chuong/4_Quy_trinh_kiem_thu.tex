\documentclass[../DoAn.tex]{subfiles}
\usepackage{longtable}
\usepackage{array}
\usepackage{booktabs}
\begin{document}

\section{Thu thập thông tin hệ thống}

Giai đoạn Thu thập thông tin được thực hiện nhằm thu thập một cách toàn diện về
 các dữ liệu liên quan đến hệ thống mục tiêu, bao gồm các công nghệ 
và nền tảng được sử dụng, các endpoint có thể truy cập từ bên ngoài, sơ đồ chức 
năng trang web cũng như 
các thành phần tiềm ẩn nguy cơ gây ra rủi ro bảo mật.

\subsection{Khảo sát hệ thống}

Hoạt động khảo sát hệ thống tập trung vào việc xác định các công nghệ, 
nền tảng và thư viện 
được sử dụng, cũng như kiểm tra các yếu tố có khả năng làm lộ thông tin hoặc 
mở rộng bề mặt tấn công của ứng dụng web. Kết quả khảo sát hệ thống được tổng 
hợp và trình bày chi tiết trong bảng dưới đây.

\begin{longtable}{|>{\centering\arraybackslash}p{1cm}|>{\centering\arraybackslash}p{2cm}|>{\centering\arraybackslash}p{2cm}|>{\raggedright\arraybackslash}p{6cm}|>{\centering\arraybackslash}p{3.2cm}|}
\caption{Kết quả khảo sát hệ thống}\\
\hline
\textbf{STT} & \textbf{Nội dung} & \textbf{Công cụ} & \textbf{Kết quả} & \textbf{Tham chiếu WSTG} \\
\hline
\endfirsthead

\hline
\textbf{STT} & \textbf{Nội dung} & \textbf{Công cụ} & \textbf{Kết quả} & \textbf{Tham chiếu WSTG} \\
\hline
\endhead

1 &
Xác định công nghệ web &
Wappalyzer &
JavaScript frameworks: RequireJS 2.3.5 \newline Video players: VideoJS \newline Font scripts: Font Awesome, Glyphicons \newline Miscellaneous: HTTP/2, Babel \newline LMS: Moodle \newline Web servers: Nginx \newline JavaScript graphics: MathJax 2.7.8 \newline Programming languages: PHP \newline JavaScript library: jQuery 3.4.1, OWL Carousel, core-js 2.6.1, YUI 3.17.2 \newline Reverse proxies: Nginx \newline UI frameworks: Bootstrap 4.3.1 &
WSTG-INFO-01 \newline WSTG-INFO-02 \newline WSTG-INFO-04 \newline WSTG-INFO-05 \newline \\
\hline

2 &
Kiểm tra các tệp siêu dữ liệu &
DevTools &
Không phát hiện thẻ meta robots; không có dấu hiệu rò rỉ dữ liệu. &
WSTG-INFO-03 \\
\hline

3 &
Phát hiện các đường dẫn web &
dirsearch, gobuster, Burp Suite &
Không phát hiện đường dẫn ẩn chứa thông tin nhạy cảm &
WSTG-INFO-02 \newline WSTG-INFO-05 \newline WSTG-INFO-07 \\
\hline

4 &
Địa chỉ IP &
host &
202.191.59.132 &
WSTG-INFO-02 \\
\hline

5 &
Xác định dịch vụ đang chạy trên các cổng mạng &
Nmap &
Các cổng dịch vụ đang mở trên mục tiêu \newline 80/TCP \newline 443/TCP &
WSTG-INFO-02 \\
\hline
\end{longtable}

\subsection{Sơ đồ chức năng trang web}

Việc phát hiện cấu trúc sitemap của hệ thống là một trong những 
kết quả thu được trong quá trình thực hiện các hạng mục kiểm thử 
thuộc nhóm Information Gathering theo khung hướng dẫn OWASP 
Web Security Testing Guide. Trên cơ sở kinh nghiệm sử dụng ứng 
dụng LMS với vai trò người dùng sinh viên, kết hợp với sự hỗ 
trợ của các công cụ quét tự động, đồ án xác định được cấu trúc 
sitemap của hệ thống mục tiêu, được trình bày chi tiết ở phần 
dưới đây.

\begin{figure}[H]
\centering
\includegraphics[width=1\linewidth]{Hinhve/Sitemap_1.png}
\caption{Sơ đồ toàn bộ chức năng Sinh viên của trang web}
\label{fig:Sitemap_1}
\end{figure}

Từ sơ đồ tổng quan các chức năng chính của hệ thống LMS dành cho sinh viên được trình bày ở Hình~\ref{fig:Sitemap_1}, có thể thấy cấu trúc website được tổ chức theo nhiều nhóm chức năng khác nhau, phục vụ các nhu cầu học tập, quản lý học phần và tương tác của người dùng. Tuy nhiên, sơ đồ tổng quan chỉ phản ánh mối quan hệ và phạm vi chức năng ở mức khái quát, chưa thể hiện đầy đủ các chức năng cụ thể mà từng trang trong hệ thống cung cấp, cũng như cách thức người dùng tương tác với từng thành phần chức năng trong hệ thống.

Do đó, để làm rõ hơn phạm vi và nội dung của từng chức năng, bảng dưới đây trình bày chi tiết các trang chính của hệ thống LMS, kèm theo mô tả và các chức năng tương ứng mà người dùng có thể thực hiện trên mỗi trang. Việc phân tích chi tiết này đóng vai trò quan trọng trong việc xác định các điểm cần kiểm thử và xây dựng các kịch bản đánh giá an toàn bảo mật trong các bước tiếp theo của đồ án.

\begin{longtable}{|>{\centering\arraybackslash}p{3.5cm}|>{\centering\arraybackslash}p{3.5cm}|>{\centering\arraybackslash}p{7cm}|}

\caption{Mô tả chi tiết chức năng của các module phần mềm Sinh viên}\\
\hline
\textbf{Trang} & \textbf{Mô tả} & \textbf{Chi tiết chức năng} \\
\hline
\endfirsthead

\hline
\textbf{Trang} & \textbf{Mô tả} & \textbf{Chi tiết chức năng} \\
\hline
\endhead

\hline
\endfoot

\hline
\endlastfoot

Đăng nhập 
& Đăng nhập hệ thống 
& 
Đăng nhập vào tài khoản cá nhân của người dùng.
\\
\hline

\multirow{4}{*}{\centering Bảng Điều khiển}
& Các khóa học được truy cập gần đây
& 
Xem thông tin các khóa học đã tham gia theo từng học kỳ.
\\
\cline{2-3}

& Tổng quan về khóa học
& 
Xem tiến trình từng khóa học.

Đánh dấu khóa học.

Xóa khóa học khỏi danh sách.
\\
\cline{2-3}

& Lịch
& 
Xem lịch.

Xuất lịch biểu.

Thêm và đồng bộ lịch từ bên ngoài vào hệ thống.
\\
\cline{2-3}
& Sự kiện sắp diễn ra
& 
Xem sự kiện sắp diễn ra.

Xuất lịch biểu.

Thêm và đồng bộ lịch từ bên ngoài vào hệ thống.
\\
\hline

\multirow{5}{*}{\centering Hồ sơ}
& Chi tiết người dùng
& 
Xem, sửa thông tin cá nhân.
\\
\cline{2-3}

& Chi tiết khóa học
& 
Xem mô tả sơ lược khóa học.
\\
\cline{2-3}

& Báo cáo
& 
Xem thông tin các phiên đăng nhập.

Xem điểm tổng quan các môn học phần.
\\
\cline{2-3}
& Hoạt động đăng nhập
& 
Xem thông tin về lần đầu truy cập trang web và lần truy cập gần nhất vào trang.
\\
\cline{2-3}

& Nội dung khác
& 
Xem thông tin các mục blog của mình.

Xem thông tin các bài viết diễn đàn.

Xem thông tin các cuộc thảo luận trong diễn đàn.

Xem thông tin kế hoạch học tập.
\\
\hline

Điểm
& 
& Xem điểm tổng quan các môn học phần. \\
\hline

\multirow{2}{*}{\centering Tin nhắn}
& Tìm kiếm
& 
Tìm người và tin nhắn.
\\
\cline{2-3}

& Nhắn tin
& 
Nhắn tin cho người khác.
\\
\hline

\multirow{3}{*}{\centering Tùy chọn}

& Tài khoản
& 
Sửa hồ sơ cá nhân.

Sửa đường dẫn trang nhà.

Sửa ngôn ngữ ưa thích.

Sửa các lựa chọn diễn đàn.

Sửa trình soạn thảo ưu tiên.

Cấu hình khóa học: Bật/tắt trình chọn hoạt động và tài nguyên.

Cài đặt ưu tiên cho lịch.

Tùy chọn tin nhắn: Chỉnh sửa quyền riêng tư, tùy chọn thông báo và thông tin chung.

Tùy chọn thông báo: Bật/tắt vô hiệu hóa thông báo.

Linked logins: Liên kết tài khoản bên ngoài thông qua dịch vụ OAuth 2.0 (HUST Login).

\\
\cline{2-3}
& Các blog
& 
Tùy chọn: Chỉnh sửa số mục blog mỗi trang.
\\
\cline{2-3}

& Điểm badges
& 
Quản lý các huy hiệu: tìm kiếm huy hiệu.

Badge preferences: Bật/tắt tự động hiện các huy hiệu đã đạt được trên trang hồ sơ.

\\
\hline

\multirow{2}{*}{\centering Khóa học}
& Semester Courses
& 
Xem thông tin các khóa học theo từng kỳ.
\\
\cline{2-3}

& A-Z Courses
& 
Xem thông tin toàn bộ các khóa học hiện có trên LMS.
\\
\hline

\multirow{2}{*}{\centering Hỗ trợ}
& Hỗ trợ Sinh viên
& 
Hướng dẫn Sinh viên sử dụng hệ thống học tập trực tuyến HUST.
\\
\cline{2-3}

& Hỗ trợ Giảng viên
& 
Hướng dẫn Giảng viên sử dụng hệ thống học tập trực tuyến HUST.
\\
\hline

Tin tức
& 
& 
Xem các thông báo mới nhất từ hệ thống.
\\
\hline

Tìm kiếm khóa học
& 
& 
Tìm kiếm các khóa học theo từ khóa.
\\
\hline

Thoát
& 
& 
Đăng xuất khỏi tài khoản người dùng.
\\
\hline

\end{longtable}

\section{Kiểm thử cấu hình và triển khai}

Trong phần này, hoạt động kiểm thử cấu hình và triển khai hệ thống 
được thực hiện dựa trên khung hướng dẫn OWASP Web Security Testing Guide 
(WSTG) nhằm đánh giá toàn 
diện mức độ an toàn của các thiết lập hệ thống, nền tảng ứng dụng và 
các cơ chế bảo vệ ở tầng hạ tầng. Nội dung kiểm thử tập trung vào việc rà soát các cấu hình tiềm ẩn nguy cơ làm lộ thông tin nhạy cảm, các chính sách bảo mật HTTP, cơ chế kiểm soát và phân quyền truy cập tài nguyên, cũng như các sai sót trong quá trình triển khai có thể bị kẻ tấn công lợi dụng, từ đó gây ra rủi ro mất an toàn thông tin cho hệ thống.

Bảng dưới đây tổng hợp kết quả kiểm thử đối với từng hạng mục 
thuộc nhóm WSTG-CONF, qua đó phản ánh mức độ tuân thủ các yêu 
cầu an toàn bảo mật của hệ thống trong công tác quản lý cấu hình 
và triển khai. Các kết quả này đồng thời chỉ ra những cơ chế đã được 
triển khai hiệu quả cũng như các vấn đề còn tồn tại cần được 
lưu ý và khắc phục.

\begin{longtable}{|>{\centering\arraybackslash}p{2cm}|>{\centering\arraybackslash}p{5cm}|>{\centering\arraybackslash}p{7cm}|}

\caption{Kết quả kiểm thử cấu hình và triển khai} \\
\hline
\textbf{ID} & \textbf{Nội dung} & \textbf{Kết luận} \\ \hline
\endfirsthead

\hline
\textbf{ID} & \textbf{Nội dung} & \textbf{Kết luận} \\ \hline
\endhead

WSTG-CONF-01 & Kiểm thử cấu hình hạ tầng mạng &
Không thể tiến hành do tài khoản kiểm thử không có quyền truy cập tài nguyên liên quan. \\ \hline

WSTG-CONF-02 & Kiểm thử cấu hình nền tảng ứng dụng &
Đạt. Không có mã debug, tệp hoặc phần mở rộng nhạy cảm nào còn sót lại trong môi trường production. Tuy nhiên, việc kiểm tra bằng cách xem mã nguồn sẽ đánh giá chính xác hơn so với kiểm thử hộp đen. \\ \hline

WSTG-CONF-03 & Kiểm thử xử lý phần mở rộng tệp đối với thông tin nhạy cảm &
Đạt. Không phát hiện tệp tin nhạy cảm hoặc dữ liệu nội bộ thông qua các phần mở rộng phổ biến (như .bak, .log, .php, .json, .zip, .env, v.v.). \\ \hline

WSTG-CONF-04 & Rà soát các bản sao lưu cũ và tệp không được tham chiếu để tìm thông tin nhạy cảm &
Không thể tiến hành do tài khoản kiểm thử không có quyền truy cập tài nguyên liên quan. \\ \hline

WSTG-CONF-05 & Liệt kê giao diện quản trị của hạ tầng và ứng dụng &
Đạt. Không tìm thấy chức năng ẩn dành cho vai trò khác trong giao diện người dùng của sinh viên. \\ \hline

WSTG-CONF-06 & Kiểm thử các phương thức HTTP &
Cho phép sử dụng các phương thức HTTP sau: GET, POST, HEAD. \\ \hline

WSTG-CONF-07 & Kiểm thử cơ chế HTTP Strict Transport Security &
Thiếu tiêu đề HSTS. \\ \hline

WSTG-CONF-08 & Kiểm thử chính sách Cross Domain của RIA &
Không thể tiến hành do tài khoản kiểm thử không có quyền truy cập tài nguyên liên quan. \\ \hline

WSTG-CONF-09 & Kiểm thử quyền truy cập tệp &
Không thể tiến hành do tài khoản kiểm thử không có quyền truy cập tài nguyên liên quan. \\ \hline

WSTG-CONF-10 & Kiểm thử khả năng chiếm quyền tên miền phụ &
Không thể tiến hành do tài khoản kiểm thử không có quyền truy cập tài nguyên liên quan. \\ \hline

WSTG-CONF-11 & Kiểm thử cấu hình lưu trữ đám mây &
Không phát hiện cấu hình sai hoặc rò rỉ dữ liệu liên quan đến các dịch vụ lưu trữ đám mây. \\ \hline

WSTG-CONF-12 & Kiểm thử chính sách bảo mật nội dung &
Thiếu tiêu đề CSP (Content Security Policy). \\ \hline

WSTG-CONF-13 & Kiểm thử nhầm lẫn đường dẫn &
Không phát hiện liên kết hoặc tệp tin riêng tư của người dùng đặc biệt bị rò rỉ. \\ \hline

WSTG-CONF-14 & Kiểm thử sai cấu hình các header bảo mật HTTP khác &
X-Frame-Options và Cache-Control bị trùng lặp có thể gây xung đột. \\ \hline

\end{longtable}


\section{Kiểm thử quản lý định danh}

Trong phần này, hoạt động kiểm thử quản lý định danh được thực hiện nhằm đánh giá mức độ an 
toàn của các cơ chế liên quan đến việc xác định và quản lý danh tính người dùng trên hệ 
thống. Nội dung kiểm thử tập trung vào các vấn đề như định nghĩa vai trò người dùng, quy 
trình đăng ký và cấp tài khoản, khả năng liệt kê hoặc suy đoán tài khoản, cũng như chính sách 
đặt tên người dùng. Việc đánh giá các hạng mục này giúp xác định những điểm yếu có thể bị lợi 
dụng để thu thập thông tin người dùng hoặc làm tiền đề cho các hình thức tấn công tiếp theo.

Bảng dưới đây Bảng dưới đây tổng hợp kết quả kiểm thử đối với từng hạng mục thuộc nhóm WSTG-IDNT, qua đó đánh giá mức độ đáp ứng của các cơ chế quản lý định danh đối với yêu cầu an toàn bảo mật, đồng thời chỉ ra những vấn đề còn tồn tại cần được lưu ý và khắc phục.

\begin{longtable}{|>{\centering\arraybackslash}p{2cm}|>{\centering\arraybackslash}p{5cm}|>{\centering\arraybackslash}p{7cm}|}
\caption{Kết quả kiểm thử quản lý định danh}\\

\hline
\textbf{ID} & \textbf{Nội dung} & \textbf{Kết luận} \\ \hline
\endfirsthead

\hline
\textbf{ID} & \textbf{Nội dung} & \textbf{Kết luận} \\ \hline
\endhead

WSTG-IDNT-01 & Kiểm thử định nghĩa vai trò &
Đạt. Không tìm được biến vai trò trong cookie, cũng như các thư mục/tệp ẩn. \\ \hline

WSTG-IDNT-02 & Kiểm thử quy trình đăng ký người dùng &
Ứng dụng không cung cấp chức năng này. \\ \hline

WSTG-IDNT-03 & Kiểm thử quy trình cấp tài khoản &
Không thể thực hiện do thiếu quyền truy cập tài nguyên và không có tài nguyên thuộc vai trò khác. \\ \hline

WSTG-IDNT-04 & Kiểm thử khả năng liệt kê tài khoản và tài khoản dễ đoán &
Tài khoản người dùng có thể đoán được, do sử dụng email của đại học làm tên đăng nhập. \\ \hline

WSTG-IDNT-05 & Kiểm thử chính sách tên người dùng yếu hoặc không được áp dụng &
Tài khoản người dùng có cấu trúc nhất quán, do sử dụng email của đại học làm tên đăng nhập. \\ \hline

\end{longtable}

\section{Kiểm thử xác thực}

Trong phần này, hoạt động kiểm thử xác thực được thực hiện nhằm đánh giá mức độ an toàn của 
các cơ chế xác thực người dùng trên hệ thống, bao gồm cơ chế 
quản lý phiên làm việc, chính sách mật khẩu và các phương thức xác thực thay thế. Nội dung 
kiểm thử tập trung vào việc xác định các điểm yếu có thể ảnh hưởng đến khả năng bảo vệ tài 
khoản người dùng, cũng như các nguy cơ rò rỉ thông tin xác thực trong quá trình sử dụng hệ 
thống.

Bảng dưới đây trình bày tổng hợp kết quả kiểm thử đối với từng hạng mục thuộc nhóm WSTG-ATHN, 
qua đó phản ánh các cơ chế xác thực đã đáp ứng yêu cầu an toàn bảo mật, các chức năng không 
được triển khai, cũng như những vấn đề còn tồn tại cần được xem xét và khắc phục.

\begin{longtable}{|>{\centering\arraybackslash}p{2cm}|>{\centering\arraybackslash}p{5.5cm}|>{\centering\arraybackslash}p{7cm}|}
\caption{Kết quả kiểm thử xác thực}\\

\hline
\textbf{ID} & \textbf{Nội dung} & \textbf{Kết luận} \\ \hline
\endfirsthead

\hline
\textbf{ID} & \textbf{Nội dung} & \textbf{Kết luận} \\ \hline
\endhead

WSTG-ATHN-01 &
Kiểm thử việc truyền thông tin xác thực qua kênh đã mã hóa &
Đã được chuyển sang mục 4.9 — Kiểm thử mật mã yếu. \\ \hline

WSTG-ATHN-02 &
Kiểm thử tài khoản mặc định &
Ứng dụng không cung cấp chức năng này. \\ \hline

WSTG-ATHN-03 &
Kiểm thử cơ chế khóa tài khoản yếu &
Không đạt. Không ghi nhận cơ chế khóa hoặc hạn chế đăng nhập sau nhiều lần xác thực thất bại. \\ \hline

WSTG-ATHN-04 &
Kiểm thử việc vượt qua sơ đồ xác thực &
Đạt. Mỗi phiên làm việc được gắn với 1 mã Session có cơ chế hết hạn. \\ \hline

WSTG-ATHN-05 &
Kiểm thử tính năng “Ghi nhớ mật khẩu” dễ bị khai thác &
Ứng dụng không cung cấp chức năng này. \\ \hline

WSTG-ATHN-06 &
Kiểm thử lỗ hổng bộ nhớ cache của trình duyệt &
Không đạt. Sau khi người dùng đăng xuất, nếu nhấn nút “Back”, trình duyệt vẫn hiển thị thông tin nhạy cảm đã xem trước đó. \\ \hline

WSTG-ATHN-07 &
Kiểm thử phương thức xác thực yếu &
Không đạt. Ứng dụng chỉ yêu cầu mật khẩu có 8 ký tự mà không có thêm yêu cầu khác về độ phức tạp (ký tự thường, ký tự hoa, số, ký tự đặc biệt). \\ \hline

WSTG-ATHN-08 &
Kiểm thử câu hỏi bảo mật dễ đoán &
Ứng dụng không cung cấp chức năng này. \\ \hline

WSTG-ATHN-09 &
Kiểm thử chức năng thay đổi hoặc đặt lại mật khẩu yếu &
Ứng dụng cho phép đặt lại mật khẩu mới trùng với mật khẩu cũ. \\ \hline

WSTG-ATHN-10 &
Kiểm thử các kênh xác thực thay thế có bảo mật kém &
Đạt. Xác thực OAuth thông qua Office 365 là an toàn. \\ \hline

WSTG-ATHN-11 &
Kiểm thử xác thực đa yếu tố (MFA) &
Ứng dụng không cung cấp chức năng này. \\ \hline

\end{longtable}

\section{Kiểm thử phân quyền}

Trong phần này, hoạt động kiểm thử phân quyền được thực hiện nhằm đánh giá mức độ an toàn 
của các cơ chế kiểm soát truy cập và phân quyền người dùng trên hệ thống. Nội dung kiểm thử 
tập trung vào việc xác định khả năng vượt qua cơ chế phân quyền, truy cập trái phép vào tài 
nguyên không được phép, leo thang đặc quyền, cũng như các lỗ hổng liên quan đến tham chiếu 
trực tiếp đối tượng không an toàn và các điểm yếu trong quá trình tích hợp OAuth.

Bảng dưới đây tổng hợp kết quả kiểm thử đối với từng hạng mục thuộc nhóm WSTG-ATHZ, qua đó 
phản ánh các cơ chế phân quyền đã đáp ứng yêu cầu an toàn bảo mật cũng như các lỗ hổng còn 
tồn tại cần được xem xét và khắc phục nhằm hạn chế nguy cơ truy cập trái phép vào hệ thống.

\begin{longtable}{|>{\centering\arraybackslash}p{2.2cm}|>{\centering\arraybackslash}p{6cm}|>{\centering\arraybackslash}p{6cm}|}
\caption{Kết quả kiểm thử phân quyền}\\
\hline
\textbf{ID} & \textbf{Nội dung} & \textbf{Kết luận} \\
\hline
\endfirsthead

\hline
\textbf{ID} & \textbf{Nội dung} & \textbf{Kết luận} \\
\hline
\endhead

WSTG-ATHZ-01 &
Kiểm thử Directory Traversal File Include &
Đạt. Không tìm thấy lỗ hổng Directory Traversal File Include. \\
\hline

WSTG-ATHZ-02 &
Kiểm thử vượt qua cơ chế phân quyền &
Đạt. Không phát hiện khả năng truy cập trái phép vào các chức năng hoặc tài nguyên ngoài quyền hạn được cấp. \\
\hline

WSTG-ATHZ-03 &
Đạt. Kiểm thử leo thang đặc quyền &
Không ghi nhận khả năng nâng cao đặc quyền trái phép. \\
\hline

WSTG-ATHZ-04 &
Kiểm thử tham chiếu trực tiếp đối tượng không an toàn &
Không đạt. Ứng dụng có lỗ hổng tham chiếu trực tiếp đối tượng không an toàn ở chức năng “linked login”. \\
\hline

WSTG-ATHZ-05 &
Kiểm thử các điểm yếu OAuth &
Đạt. Xác thực OAuth thông qua Office 365 là an toàn. \\
\hline
\end{longtable}

\section{Kiểm thử quản lý phiên}

Trong phần này, hoạt động kiểm thử quản lý phiên được thực hiện nhằm đánh giá mức 
độ an toàn của các cơ chế quản lý phiên làm việc của người dùng trên hệ thống. Nội 
dung kiểm thử tập trung vào việc phân tích cách thức tạo và quản lý phiên, cấu hình và thuộc tính của cookie, cơ chế kết thúc và hết hạn phiên, cũng như các nguy cơ liên quan đến chiếm đoạt hoặc lạm dụng phiên làm việc, từ đó xác định mức độ tuân thủ các khuyến nghị bảo mật hiện hành.

Bảng dưới đây tổng hợp kết quả kiểm thử đối với từng hạng mục thuộc nhóm WSTG-SESS, qua đó phản ánh các cơ chế quản lý phiên đã đáp ứng yêu cầu an toàn bảo mật, các vấn đề cấu hình chưa phù hợp và những rủi ro tiềm ẩn cần được xem xét và khắc phục trong các phần tiếp theo.

\begin{longtable}{|>{\centering\arraybackslash}p{2.2cm}|>{\centering\arraybackslash}p{5cm}|>{\centering\arraybackslash}p{7cm}|}
\caption{Kết quả kiểm thử quản lý phiên}\\

\hline
\textbf{ID} & \textbf{Nội dung} & \textbf{Kết luận} \\ \hline
\endfirsthead

\hline
\textbf{ID} & \textbf{Nội dung} & \textbf{Kết luận} \\ \hline
\endhead

WSTG-SESS-01 &
Kiểm tra sơ đồ quản lý phiên &
Đạt. Cookie định danh người dùng MoodleSession có đủ tính ngẫu nhiên, khó bị dò đoán, phân tích. \\ \hline

WSTG-SESS-02 &
Kiểm tra các thuộc tính của cookie &
Không đạt. Cookie không được cấu hình thuộc tính HttpOnly. \\ \hline

WSTG-SESS-03 &
Kiểm tra lỗ hổng cố định phiên &
Đạt. Cơ chế phân quyền phiên có thời gian hiệu lực và chức năng kết thúc phiên để chấm dứt phiên bất cứ khi nào người dùng đăng xuất. \\ \hline

WSTG-SESS-04 &
Kiểm tra các biến phiên bị lộ &
Đạt. Cơ chế Cache-Control được bảo mật, tuy nhiên header Expires nên để là “0” thay vì để trống. \\ \hline

WSTG-SESS-05 &
Kiểm tra lỗ hổng CSRF &
Đạt. Không phát hiện lỗ hổng Cross-Site Request Forgery (CSRF) trong quá trình kiểm thử. \\ \hline

WSTG-SESS-06 &
Kiểm tra chức năng đăng xuất &
Đạt. Thời gian chờ phiên và cơ chế hủy phiên sau khi đăng xuất được thực hiện đúng cách. \\ \hline

WSTG-SESS-07 &
Kiểm tra cơ chế hết hạn phiên &
Đạt. Thời gian hết hạn phiên khoảng 4 giờ hoạt động đúng cách. \\ \hline

WSTG-SESS-08 &
Kiểm tra lỗ hổng Session Puzzling &
Đat. Không phát hiện lỗ hổng Session Puzzling trong phạm vi kiểm thử. \\ \hline

WSTG-SESS-09 &
Kiểm tra lỗ hổng chiếm đoạt phiên &
Cookie không có cờ HttpOnly, điều này có thể dẫn đến lỗ hổng chiếm đoạt phiên. \\ \hline

WSTG-SESS-10 &
Kiểm tra JWT (JSON Web Tokens) &
Không tiến hành kiểm thử do ứng dụng không sử dụng JWT. \\ \hline

WSTG-SESS-11 &
Kiểm tra phiên đăng nhập đồng thời &
 \\ \hline

\end{longtable}

\section{Kiểm thử xác thực đầu vào}

Trong phần này, hoạt động kiểm thử xác thực đầu vào được thực hiện nhằm đánh giá khả năng 
kiểm soát và xử lý dữ liệu do người dùng cung cấp trước khi được đưa vào các thành phần xử 
lý phía máy chủ. Nội dung kiểm thử tập trung vào việc xác định các lỗ hổng phổ biến phát sinh 
từ việc thiếu hoặc thực hiện không đầy đủ cơ chế kiểm tra dữ liệu đầu vào, bao gồm các dạng 
tấn công như XSS, Injection, giả mạo yêu cầu HTTP và các kỹ thuật khai thác liên quan đến xử 
lý tham số.

Bảng dưới đây trình bày tổng hợp kết quả kiểm thử đối với từng hạng mục thuộc nhóm WSTG-INPV, 
qua đó phản ánh các cơ chế kiểm soát dữ liệu đầu vào đã đáp ứng yêu cầu an toàn bảo mật, đồng 
thời chỉ ra các trường hợp chưa đạt và những lỗ hổng còn tồn tại có thể bị khai thác nếu không 
được khắc phục kịp thời.

\begin{longtable}{|>{\centering\arraybackslash}p{2cm}|>{\centering\arraybackslash}p{4cm}|>{\centering\arraybackslash}p{8cm}|}
\caption{Kết quả kiểm thử xác thực đầu vào}\\
\hline
\textbf{ID} & \textbf{Nội dung} & \textbf{Kết luận} \\
\hline
\endfirsthead

\hline
\textbf{ID} & \textbf{Nội dung} & \textbf{Kết luận} \\
\hline
\endhead

WSTG-INPV-01 & Kiểm thử XSS phản chiếu & Đạt. Không phát hiện lỗ hổng XSS phản chiếu. \\ \hline

WSTG-INPV-02 & Kiểm thử XSS lưu trữ & 
Không đạt. Chức năng “Tạo sự kiện mới”, “Chỉnh sửa hồ sơ cá nhân” có lưu trữ và hiển thị lại dữ liệu phía client. \\ \hline

WSTG-INPV-03 & Kiểm thử giả mạo phương thức HTTP & Đạt. Không phát hiện lỗ hổng giả mạo phương thức HTTP. \\ \hline

WSTG-INPV-04 & Kiểm thử ô nhiễm tham số HTTP & Đạt. Không phát hiện lỗ hổng ô nhiễm tham số HTTP. \\ \hline

WSTG-INPV-05 & Kiểm thử SQL Injection & Không đạt. Ứng dụng có tồn tại lỗ hổng SQL Injection ở Dashboard. \\ \hline

WSTG-INPV-06 & Kiểm thử LDAP Injection & Đạt. Không phát hiện lỗ hổng LDAP Injection. \\ \hline

WSTG-INPV-07 & Kiểm thử XML Injection & Đạt. Không phát hiện lỗ hổng XML Injection. \\ \hline

WSTG-INPV-08 & Kiểm thử SSI Injection & Đạt. Không phát hiện lỗ hổng SSI Injection. \\ \hline

WSTG-INPV-09 & Kiểm thử Xpath Injection & Đạt. Không phát hiện lỗ hổng Xpath Injection. \\ \hline

WSTG-INPV-10 & Kiểm thử IMAP SMTP Injection & Đạt. Không phát hiện lỗ hổng IMAP SMTP Injection. \\ \hline

WSTG-INPV-11 & Kiểm thử Code Injection & Đạt. Không phát hiện lỗ hổng Code Injection. \\ \hline

WSTG-INPV-12 & Kiểm thử File Inclusion & Đạt. Không phát hiện lỗ hổng File Inclusion. \\ \hline

WSTG-INPV-13 & Kiểm thử Format String Injection & Đạt. Không phát hiện lỗ hổng Format String Injection. \\ \hline

WSTG-INPV-14 & Kiểm thử lỗ hổng ẩn & Đạt. Không phát hiện lỗ hổng ẩn. \\ \hline

WSTG-INPV-15 & Kiểm thử HTTP Splitting / Smuggling & Không đạt. Ứng dụng có tồn tại lỗ hỗng HTTP Smuggling có thể lấy được request của người khác bao gồm cả phiên đăng nhập  \\ \hline

WSTG-INPV-16 & Kiểm thử xử lý yêu cầu HTTP đến & Đạt. Hệ thống không phát sinh các HTTP request không cần thiết hoặc đáng ngờ. \\ \hline

WSTG-INPV-17 & Kiểm thử Host Header Injection & Đạt. Không phát hiện lỗ hổng Host Header Injection. \\ \hline

WSTG-INPV-18 & Kiểm thử Server-side Template Injection & Đạt. Không phát hiện lỗ hổng Server-side Template Injection. \\ \hline

WSTG-INPV-19 & Kiểm thử Server-Side Request Forgery & Đạt. Không phát hiện lỗ hổng Server-Side Request Forgery. \\ \hline

WSTG-INPV-20 & Kiểm thử Mass Assignment & Đạt. Không thể chỉnh sửa các trường dữ liệu ngoài phạm vi cho phép. \\ \hline

\end{longtable}

\section{Kiểm thử xử lý lỗi}

Trong phần này, hoạt động kiểm thử xử lý lỗi được thực hiện nhằm đánh giá cách thức hệ thống phản hồi và xử lý các tình huống 
lỗi phát sinh trong quá trình vận hành. Nội dung kiểm thử tập trung vào việc xác định khả năng làm lộ thông tin nội bộ thông 
qua thông báo lỗi không phù hợp, stack trace hoặc các phản hồi chi tiết từ phía máy chủ, vốn có thể bị kẻ tấn công lợi dụng để 
thu thập thông tin phục vụ cho các bước tấn công tiếp theo.

Bảng dưới đây tổng hợp kết quả kiểm thử đối với từng hạng mục thuộc nhóm WSTG-ERRH, 
qua đó phản ánh mức độ an toàn của cơ chế xử lý lỗi và khả năng kiểm soát 
thông tin trả về cho người dùng khi xảy ra lỗi, đồng thời đánh giá nguy cơ lộ lọt thông tin nhạy cảm có thể bị kẻ tấn công lợi dụng.

\begin{longtable}{|>{\centering\arraybackslash}p{3.5cm}|>{\centering\arraybackslash}p{3cm}|>{\centering\arraybackslash}p{7.5cm}|}
\caption{Kết quả kiểm thử xử lý lỗi}\\
\hline
\textbf{ID} & \textbf{Nội dung} & \textbf{Kết luận} \\
\hline
\endfirsthead

\hline
\textbf{ID} & \textbf{Nội dung} & \textbf{Kết luận} \\
\hline
\endhead

WSTG-ERRH-01 & Kiểm thử xử lý lỗi không đúng cách & Đạt. Không ghi nhận thông báo lỗi chi tiết hoặc rò rỉ thông tin nội bộ; các phản hồi lỗi được xử lý thống nhất và không để lộ dữ liệu nhạy cảm. \\ \hline

WSTG-ERRH-02 & Kiểm thử việc lộ Stack trace & Nội dung này đã được hợp nhất vào mục WSTG-ERRH-01 Kiểm thử xử lý lỗi không đúng cách. \\ \hline

\end{longtable}

\section{Kiểm thử mật mã yếu}

Trong phần này, hoạt động kiểm thử các cơ chế mật mã được thực hiện nhằm đánh giá mức độ an toàn của 
các giải pháp mã hóa được sử dụng trong quá trình truyền tải và xử lý dữ liệu trên hệ thống. Nội dung 
kiểm thử tập trung vào việc xác định các điểm yếu liên quan đến lớp bảo mật truyền tải, việc sử dụng 
các thuật toán hoặc bộ mã hóa không còn an toàn, cũng như nguy cơ rò rỉ thông tin nhạy cảm khi dữ 
liệu được truyền qua các kênh không được mã hóa đầy đủ.

Bảng dưới đây trình bày tổng hợp kết quả kiểm thử đối với từng hạng mục thuộc nhóm WSTG-CRYP, qua đó 
phản ánh mức độ tuân thủ các yêu cầu bảo mật về mật mã của hệ thống, đồng thời chỉ ra các trường hợp 
chưa đạt do tồn tại các cấu hình hoặc cơ chế mã hóa yếu cần được xem xét và khắc phục.

\begin{longtable}{|>{\centering\arraybackslash}p{3.5cm}|>{\centering\arraybackslash}p{4cm}|>{\centering\arraybackslash}p{6.5cm}|}
\caption{Kết quả kiểm thử mật mã yếu}\\
\hline
\textbf{ID} & \textbf{Nội dung} & \textbf{Kết luận} \\
\hline
\endfirsthead

\hline
\textbf{ID} & \textbf{Nội dung} & \textbf{Kết luận} \\
\hline
\endhead

WSTG-CRYP-01 & Kiểm thử bảo mật lớp truyền tải yếu & Không đạt. Giao thức TLS 1.2 sử dụng các bộ mã hóa yếu. \\ \hline

WSTG-CRYP-02 & Kiểm thử lỗ hổng Padding Oracle & Đạt. Không phát hiện lỗ hổng Padding Oracle trong quá trình kiểm thử. \\ \hline

WSTG-CRYP-03 & Kiểm thử thông tin nhạy cảm được gửi qua kênh không được mã hóa & Đạt. Các kênh truyền đảm bảo mức độ riêng tư và an toàn \\ \hline

WSTG-CRYP-04 & Kiểm thử mã hóa yếu & Không đạt. Giao thức TLS 1.2 sử dụng các bộ mã hóa yếu. \\ \hline

\end{longtable}


\section{Kiểm thử phía máy người dùng}

Trong phần này, hoạt động kiểm thử phía máy người dùng được thực hiện nhằm đánh giá mức 
độ an toàn của các cơ chế xử lý và hiển thị dữ liệu trên trình duyệt. Nội dung kiểm thử tập trung vào 
việc xác định các lỗ hổng phát sinh từ việc thực thi mã phía client, thao tác với DOM, lưu trữ dữ 
liệu trên trình duyệt, cũng như các cơ chế bảo vệ liên quan đến chia sẻ tài nguyên và tương tác giữa 
các thành phần phía client.

Bảng dưới đây tổng hợp kết quả kiểm thử đối với từng hạng mục thuộc nhóm WSTG-CLNT, qua đó phản ánh các chức năng phía client đã đáp ứng yêu cầu an toàn bảo mật, các trường hợp chưa đạt hoặc còn tồn tại rủi ro, cũng như những hạng mục chưa thể đánh giá đầy đủ do giới hạn của mô hình kiểm thử hộp đen.
Kết quả này sẽ là cơ sở để đề xuất các biện pháp khắc phục phù hợp ở phần tiếp theo.

\begin{longtable}{|>{\centering\arraybackslash}p{3.5cm}|>{\centering\arraybackslash}p{4.5cm}|>{\centering\arraybackslash}p{6cm}|}
\caption{Kết quả kiểm thử phía máy người dùng}\\
\hline
\textbf{ID} & \textbf{Nội dung} & \textbf{Kết luận} \\
\hline
\endfirsthead

\hline
\textbf{ID} & \textbf{Nội dung} & \textbf{Kết luận} \\
\hline
\endhead

WSTG-CLNT-01 & Kiểm thử lỗ hổng Cross-Site Scripting dựa trên DOM (DOM-Based XSS) & Đạt. Không phát hiện lỗ hổng DOM-Based XSS. \\ \hline

WSTG-CLNT-02 & Kiểm thử khả năng thực thi mã JavaScript phía trình duyệt & Không đạt. Chức năng “Tạo sự kiện mới”, “Chỉnh sửa hồ sơ cá nhân” có lưu trữ và hiển thị lại dữ liệu phía client. \\ \hline

WSTG-CLNT-03 & Kiểm thử lỗ hổng chèn mã HTML & Đạt. Không phát hiện lỗ hổng chèn mã HTML. \\ \hline

WSTG-CLNT-04 & Kiểm thử lỗ hổng chuyển hướng URL phía client & Đạt. Không phát hiện lỗ hổng chuyển hướng URL phía client. \\ \hline

WSTG-CLNT-05 & Kiểm thử lỗ hổng chèn mã CSS & Đạt. Không phát hiện lỗ hổng chèn mã CSS. \\ \hline

WSTG-CLNT-06 & Kiểm thử thao túng tài nguyên phía client & Đạt. Không phát hiện lỗ hổng thao túng tài nguyên phía client. \\ \hline

WSTG-CLNT-07 & Kiểm thử chia sẻ tài nguyên chéo nguồn (Cross-Origin Resource Sharing - CORS) & Đạt. Cấu hình CORS không phát hiện rủi ro bảo mật trong quá trình kiểm thử. \\ \hline

WSTG-CLNT-09 & Kiểm thử lỗ hổng Clickjacking & Đạt. Cơ chế X-Frame-Options: SAMEORIGIN được cấu hình đúng và không phát hiện lỗ hổng clickjacking. \\ \hline

WSTG-CLNT-12 & Kiểm thử lưu trữ dữ liệu trên trình duyệt &
Đạt. Không phát hiện thông tin nhạy cảm được lưu trữ trong bộ nhớ trình duyệt của người dùng. \\ \hline

WSTG-CLNT-13 & Kiểm thử Cross-Site Script Inclusion (XSSI) &
Khó kiểm thử trong mô hình kiểm thử xâm nhập hộp đen \\ \hline

WSTG-CLNT-14 & Kiểm thử tấn công Reverse Tabnabbing &
Khó kiểm thử trong mô hình kiểm thử xâm nhập hộp đen do không có quyền truy cập cần thiết vào tài nguyên. \\ \hline


\end{longtable}

% \section{Kiểm thử API}

% Trong phần này, hoạt động kiểm thử API được thực hiện nhằm đánh giá mức độ an toàn của các giao diện 
% lập trình ứng dụng được hệ thống sử dụng để trao đổi và xử lý dữ liệu. Nội dung kiểm thử tập trung 
% vào việc thu thập thông tin liên quan đến các endpoint API, cơ chế xác thực và phân quyền, cũng như 
% khả năng kiểm soát truy cập đối tượng nhằm phát hiện các lỗ hổng phổ biến trong API.

% Bảng dưới đây trình bày tổng hợp kết quả kiểm thử đối với từng hạng mục thuộc nhóm WSTG-APIT, qua đó 
% phản ánh mức độ an toàn của các API được triển khai trên hệ thống, đồng thời chỉ ra các vấn đề liên 
% quan đến kiểm soát truy cập và quản lý dữ liệu cần được xem xét trong quá trình đánh giá tổng thể.

% \begin{longtable}{|>{\centering\arraybackslash}p{3.5cm}|p{7cm}|p{3.5cm}|}
% \hline
% \textbf{ID} & \textbf{Nội dung} & \textbf{Kết luận} \\
% \hline
% \endfirsthead

% \hline
% \textbf{ID} & \textbf{Nội dung} & \textbf{Kết luận} \\
% \hline
% \endhead

% WSTG-APIT-01 & Thu thập thông tin API & Kết quả được ghi nhận trong giai đoạn thu thập thông tin. \\ \hline

% WSTG-APIT-02 & Lỗi kiểm soát truy cập theo đối tượng trong API & Đạt. \\ \hline

% \caption{Kết quả kiểm thử API}
% \end{longtable}

\end{document}
