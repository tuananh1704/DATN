\documentclass[../DoAn.tex]{subfiles}
\usepackage{longtable}
\usepackage{array}
\usepackage{booktabs}

\begin{document}

Trong mục này, nội dung sẽ trình bày mục đích của từng nội dung kiểm thử thuộc 10 nhóm trong quy trình kiểm thử xâm nhập chủ động.

\setcounter{section}{0}
\renewcommand{\thesection}{\arabic{section}}
\titleformat{\section}
    [hang]
    {\bfseries}
    {\arabic{section}.\ \ \ \ }
    {0pt}
    {}
    []
\titlespacing{\section}{0pt}{\parskip}{0.5\parskip}

\section{Thu thập thông tin}
\begin{longtable}{|>{\centering\arraybackslash}p{1.3cm}|p{4.5cm}|p{7.7cm}|}

\caption{Mục đích của nội dung thu thập thông tin}\\
\hline
\textbf{Mã} & \textbf{Nội dung kiểm thử} & \textbf{Mục tiêu} \\
\hline
\endfirsthead

\hline
\textbf{Mã} & \textbf{Nội dung kiểm thử} & \textbf{Mục tiêu} \\
\hline
\endhead

WSTG-INFO-01 &
Thực hiện do thám qua công cụ tìm kiếm để phát hiện rò rỉ thông tin &
Xác định các thông tin nhạy cảm liên quan đến thiết kế và cấu hình của ứng dụng, hệ thống hoặc tổ chức bị lộ trực tiếp trên website của tổ chức hoặc gián tiếp thông qua các dịch vụ bên thứ ba.
\\ \hline

WSTG-INFO-02 &
Dấu vân tay máy chủ web &
Xác định loại và phiên bản máy chủ web đang vận hành nhằm hỗ trợ quá trình tra cứu và phát hiện các lỗ hổng đã được công bố.
\\ \hline

WSTG-INFO-03 &
Rà soát tệp siêu dữ liệu của máy chủ web để phát hiện rò rỉ thông tin &
Xác định các đường dẫn hoặc chức năng bị ẩn hay được làm mờ thông qua phân tích các tệp siêu dữ liệu, đồng thời trích xuất và ánh xạ các thông tin giúp hiểu rõ hơn về hệ thống mục tiêu.
\\ \hline

WSTG-INFO-04 &
Liệt kê các ứng dụng trên máy chủ web &
Liệt kê các ứng dụng nằm trong phạm vi kiểm thử hiện đang tồn tại trên máy chủ web nhằm xác định đầy đủ bề mặt tấn công.
\\ \hline

WSTG-INFO-05 &
Rà soát nội dung trang web để phát hiện rò rỉ thông tin &
Phân tích bình luận, metadata, nội dung phản hồi của các cơ chế chuyển hướng, mã JavaScript phía client, source map và các tệp gỡ lỗi frontend nhằm phát hiện các thông tin bị rò rỉ và hỗ trợ hiểu rõ hơn về cấu trúc cũng như logic của ứng dụng.
\\ \hline

WSTG-INFO-06 &
Xác định các điểm vào của ứng dụng &
Xác định các điểm đầu vào và các vị trí có khả năng xảy ra tiêm nhiễm hoặc chèn mã thông qua việc phân tích các request và response của ứng dụng.
\\ \hline

WSTG-INFO-07 &
Lập bản đồ các luồng thực thi trong ứng dụng &
Lập bản đồ tổng thể ứng dụng mục tiêu và nắm bắt các luồng nghiệp vụ, chức năng chính phục vụ cho các bước kiểm thử chuyên sâu tiếp theo.
\\ \hline

WSTG-INFO-08 &
Dấu vân tay framework ứng dụng web &
Xác định các framework và thành phần công nghệ được sử dụng trong ứng dụng web nhằm hỗ trợ đánh giá rủi ro và tra cứu lỗ hổng liên quan.
\\ \hline

WSTG-INFO-09 &
Dấu vân tay ứng dụng web &
Xác định đặc điểm, công nghệ và các thành phần đặc thù của ứng dụng web để phục vụ quá trình đánh giá tổng thể bề mặt tấn công.
\\ \hline

WSTG-INFO-10 &
Lập bản đồ kiến trúc ứng dụng &
Phân tích kiến trúc tổng thể của ứng dụng và các công nghệ đang được sử dụng nhằm hiểu rõ cách thức triển khai và mối quan hệ giữa các thành phần trong hệ thống.
\\ \hline


\end{longtable}

\section{Kiểm thử cấu hình và quản lý triển khai}
\begin{longtable}{|>{\centering\arraybackslash}p{1.3cm}|p{4.5cm}|p{7.7cm}|}

\caption{Mục đích của nội dung kiểm thử cấu hình và quản lý triển khai}\\
\hline
\textbf{Mã} & \textbf{Nội dung kiểm thử} & \textbf{Mục tiêu} \\
\hline
\endfirsthead

\hline
\textbf{Mã} & \textbf{Nội dung kiểm thử} & \textbf{Mục tiêu} \\
\hline
\endhead

WSTG-CONF-01 &
Kiểm thử cấu hình hạ tầng mạng &
Rà soát các cấu hình của ứng dụng được triển khai trên toàn bộ hạ tầng mạng và xác minh rằng các cấu hình này không tồn tại lỗ hổng bảo mật. Đồng thời đánh giá mức độ an toàn của các framework và hệ thống đang sử dụng, bảo đảm chúng không dễ bị khai thác do phần mềm không được cập nhật hoặc sử dụng cấu hình và thông tin xác thực mặc định.
\\ \hline

WSTG-CONF-02 &
Kiểm thử cấu hình nền tảng ứng dụng &
Đảm bảo các tệp và cấu hình mặc định hoặc đã được công bố rộng rãi đã được loại bỏ, đồng thời xác minh rằng không tồn tại mã gỡ lỗi hoặc các tiện ích mở rộng trong môi trường sản xuất. Ngoài ra, đánh giá các cơ chế ghi log đang được áp dụng cho ứng dụng.
\\ \hline

WSTG-CONF-03 &
Kiểm thử xử lý phần mở rộng tệp chứa thông tin nhạy cảm &
Thực hiện thử nghiệm brute-force các phần mở rộng tệp có khả năng chứa script, thông tin xác thực hoặc dữ liệu nhạy cảm, đồng thời xác minh rằng không tồn tại cơ chế bypass của framework hoặc hệ thống đối với các quy tắc bảo mật đã được thiết lập.
\\ \hline

WSTG-CONF-04 &
Rà soát các bản sao lưu cũ và tệp tham chiếu để phát hiện thông tin nhạy cảm &
Xác định và phân tích các tệp không còn được tham chiếu hoặc sử dụng nhưng vẫn tồn tại trên hệ thống, có khả năng chứa thông tin nhạy cảm.
\\ \hline

WSTG-CONF-05 &
Liệt kê hạ tầng và các giao diện quản trị ứng dụng &
Xác định các giao diện quản trị và chức năng quản trị bị ẩn nhằm đánh giá nguy cơ truy cập trái phép.
\\ \hline

WSTG-CONF-06 &
Kiểm thử các phương thức HTTP &
Liệt kê các phương thức HTTP được hỗ trợ, kiểm tra khả năng vượt qua kiểm soát truy cập và đánh giá các kỹ thuật ghi đè phương thức HTTP.
\\ \hline

WSTG-CONF-07 &
Kiểm thử cơ chế bảo mật truyền tải HTTP nghiêm ngặt &
Rà soát tiêu đề HSTS và đánh giá tính hợp lệ cũng như mức độ hiệu quả của cơ chế bảo mật này.
\\ \hline

WSTG-CONF-08 &
Kiểm thử chính sách cross-domain của RIA &
Đánh giá các chính sách cross-domain được áp dụng cho các ứng dụng RIA nhằm phát hiện các cấu hình sai có thể dẫn đến rủi ro bảo mật.
\\ \hline

WSTG-CONF-09 &
Kiểm thử phân quyền truy cập tệp &
Rà soát và xác định các quyền truy cập tệp được cấu hình không phù hợp hoặc vượt quá mức cần thiết.
\\ \hline

WSTG-CONF-10 &
Kiểm thử khả năng chiếm quyền subdomain &
Liệt kê tất cả các domain và subdomain (hiện tại và trước đây), đồng thời xác định các domain bị bỏ quên hoặc cấu hình sai có thể dẫn đến nguy cơ chiếm quyền.
\\ \hline

WSTG-CONF-11 &
Kiểm thử lưu trữ đám mây &
Đánh giá việc cấu hình kiểm soát truy cập đối với các dịch vụ lưu trữ đám mây nhằm đảm bảo các biện pháp bảo mật được thiết lập đầy đủ.
\\ \hline

WSTG-CONF-12 &
Kiểm thử chính sách bảo mật nội dung &
Rà soát tiêu đề Content-Security-Policy hoặc thẻ meta tương ứng để phát hiện các cấu hình sai hoặc chưa đầy đủ.
\\ \hline

WSTG-CONF-13 &
Kiểm thử nhầm lẫn đường dẫn &
Xác minh rằng các đường dẫn của ứng dụng được cấu hình chính xác và không gây ra nhầm lẫn trong quá trình xử lý.
\\ \hline

WSTG-CONF-14 &
Kiểm thử các cấu hình sai liên quan đến HTTP Security Header &
Xác định các tiêu đề bảo mật HTTP được cấu hình không đúng, đánh giá tác động của các cấu hình sai này và xác minh việc triển khai đầy đủ các tiêu đề bảo mật bắt buộc.
\\ \hline

\end{longtable}


\section{Kiểm thử quản lý định danh}
\begin{longtable}{|>{\centering\arraybackslash}p{1.3cm}|p{4.5cm}|p{7.7cm}|}

\caption{Mục đích của nội dung kiểm thử quản lý định danh}\\
\hline
\textbf{Mã} & \textbf{Nội dung kiểm thử} & \textbf{Mục tiêu} \\
\hline
\endfirsthead

\hline
\textbf{Mã} & \textbf{Nội dung kiểm thử} & \textbf{Mục tiêu} \\
\hline
\endhead

WSTG-IDNT-01 &
Kiểm thử định nghĩa vai trò &
Xác định và ghi nhận các vai trò được sử dụng trong ứng dụng, đồng thời thử nghiệm khả năng chuyển đổi, thay đổi hoặc truy cập trái phép sang các vai trò khác. Bên cạnh đó, đánh giá mức độ chi tiết của các vai trò và sự phù hợp của các quyền được cấp.
\\ \hline

WSTG-IDNT-02 &
Kiểm thử quy trình đăng ký người dùng &
Xác minh rằng các yêu cầu định danh trong quy trình đăng ký người dùng phù hợp với yêu cầu nghiệp vụ và yêu cầu bảo mật, đồng thời đánh giá tính hợp lệ và đầy đủ của quy trình đăng ký.
\\ \hline

WSTG-IDNT-03 &
Kiểm thử quy trình cấp phát tài khoản &
Xác định các tài khoản có khả năng tạo hoặc cấp phát các tài khoản khác, cũng như loại tài khoản mà chúng có thể cấp phát, nhằm đánh giá nguy cơ lạm dụng quyền.
\\ \hline

WSTG-IDNT-04 &
Kiểm thử liệt kê tài khoản và suy đoán tài khoản người dùng &
Rà soát các quy trình liên quan đến việc định danh người dùng như đăng ký, đăng nhập, đồng thời xác định khả năng liệt kê tài khoản thông qua phân tích phản hồi từ hệ thống.
\\ \hline

WSTG-IDNT-05 &
Kiểm thử chính sách tên người dùng yếu hoặc không được áp dụng chặt chẽ &
Đánh giá việc sử dụng cấu trúc tên tài khoản nhất quán có thể dẫn đến nguy cơ liệt kê tài khoản, đồng thời xác định liệu các thông báo lỗi của ứng dụng có cho phép suy đoán hoặc liệt kê tài khoản người dùng hay không.
\\ \hline

\end{longtable}

\section{Kiểm thử xác thực}
\begin{longtable}{|>{\centering\arraybackslash}p{1.3cm}|p{4.5cm}|p{7.7cm}|}

\caption{Mục đích của nội dung kiểm thử xác thực}\\
\hline
\textbf{Mã} & \textbf{Nội dung kiểm thử} & \textbf{Mục tiêu} \\
\hline
\endfirsthead

\hline
\textbf{Mã} & \textbf{Nội dung kiểm thử} & \textbf{Mục tiêu} \\
\hline
\endhead

WSTG-ATHN-01 &
Kiểm thử việc truyền thông tin xác thực qua kênh mã hóa &
Xác định liệu thông tin xác thực của người dùng có được truyền tải thông qua các kênh truyền thông được mã hóa an toàn nhằm ngăn chặn nguy cơ nghe lén hoặc đánh cắp dữ liệu hay không.
\\ \hline

WSTG-ATHN-02 &
Kiểm thử thông tin xác thực mặc định &
Xác định sự tồn tại của các tài khoản người dùng sử dụng mật khẩu mặc định, đồng thời đánh giá việc tạo tài khoản mới có sử dụng mật khẩu yếu hoặc dễ đoán hay không.
\\ \hline

WSTG-ATHN-03 &
Kiểm thử cơ chế khóa tài khoản yếu &
Đánh giá khả năng của cơ chế khóa tài khoản trong việc giảm thiểu các cuộc tấn công brute-force đoán mật khẩu, đồng thời xác định mức độ an toàn của cơ chế mở khóa tài khoản trước các hành vi truy cập trái phép.
\\ \hline

WSTG-ATHN-04 &
Kiểm thử khả năng bỏ qua cơ chế xác thực &
Xác minh rằng cơ chế xác thực được áp dụng đầy đủ và nhất quán đối với tất cả các dịch vụ và chức năng yêu cầu xác thực.
\\ \hline

WSTG-ATHN-05 &
Kiểm thử chức năng ghi nhớ mật khẩu không an toàn &
Xác thực rằng phiên đăng nhập được tạo ra được quản lý một cách an toàn và không làm lộ hoặc gây nguy hiểm cho thông tin xác thực của người dùng.
\\ \hline

WSTG-ATHN-06 &
Kiểm thử điểm yếu liên quan đến bộ nhớ đệm trình duyệt &
Rà soát việc ứng dụng có lưu trữ thông tin nhạy cảm phía client hay không, đồng thời đánh giá khả năng truy cập dữ liệu mà không cần xác thực hoặc phân quyền hợp lệ.
\\ \hline

WSTG-ATHN-07 &
Kiểm thử các phương thức xác thực yếu &
Đánh giá khả năng chống chịu của ứng dụng trước các cuộc tấn công brute-force bằng cách xem xét các yêu cầu về độ dài, độ phức tạp, khả năng tái sử dụng và vòng đời của mật khẩu.
\\ \hline

WSTG-ATHN-08 &
Kiểm thử câu hỏi bảo mật yếu &
Đánh giá mức độ phức tạp và tính dễ đoán của các câu hỏi bảo mật, đồng thời xem xét khả năng suy đoán hoặc brute-force câu trả lời của người dùng.
\\ \hline

WSTG-ATHN-09 &
Kiểm thử chức năng thay đổi hoặc đặt lại mật khẩu yếu &
Xác định liệu các chức năng thay đổi hoặc đặt lại mật khẩu có tồn tại điểm yếu có thể bị lợi dụng để chiếm quyền tài khoản người dùng hay không.
\\ \hline

WSTG-ATHN-10 &
Kiểm thử xác thực yếu trên kênh thay thế &
Xác định các kênh xác thực thay thế đang được sử dụng và đánh giá các biện pháp bảo mật được áp dụng, đồng thời kiểm tra khả năng tồn tại các cơ chế bypass trên các kênh này.
\\ \hline

WSTG-ATHN-11 &
Kiểm thử xác thực đa yếu tố &
Xác định loại cơ chế xác thực đa yếu tố được ứng dụng triển khai, đánh giá mức độ an toàn và tính vững chắc của việc triển khai MFA, đồng thời thử nghiệm khả năng vượt qua cơ chế xác thực đa yếu tố.
\\ \hline
\end{longtable}

\section{Kiểm thử phân quyền}
\begin{longtable}{|>{\centering\arraybackslash}p{1.3cm}|p{4.5cm}|p{7.7cm}|}

\caption{Mục đích của nội dung kiểm thử phân quyền}\\
\hline
\textbf{Mã} & \textbf{Nội dung kiểm thử} & \textbf{Mục tiêu} \\
\hline
\endfirsthead

\hline
\textbf{Mã} & \textbf{Nội dung kiểm thử} & \textbf{Mục tiêu} \\
\hline
\endhead

WSTG-ATHZ-01 &
Kiểm thử traversal thư mục và file include &
Xác định các điểm có khả năng xảy ra chèn tham số liên quan đến traversal đường dẫn, đồng thời đánh giá các kỹ thuật vượt qua cơ chế bảo vệ và mức độ ảnh hưởng của lỗ hổng path traversal nếu tồn tại.
\\ \hline

WSTG-ATHZ-02 &
Kiểm thử khả năng bỏ qua cơ chế phân quyền &
Đánh giá khả năng truy cập trái phép theo chiều ngang hoặc chiều dọc nhằm xác định việc thực thi phân quyền của ứng dụng có được áp dụng đầy đủ và chính xác hay không.
\\ \hline

WSTG-ATHZ-03 &
Kiểm thử leo thang đặc quyền &
Xác định các điểm có khả năng xảy ra thao túng đặc quyền thông qua việc chèn tham số hoặc thay đổi dữ liệu, đồng thời thực hiện fuzzing hoặc các kỹ thuật khác nhằm đánh giá khả năng vượt qua các biện pháp bảo mật.
\\ \hline

WSTG-ATHZ-04 &
Kiểm thử tham chiếu đối tượng trực tiếp không an toàn (IDOR) &
Xác định các vị trí sử dụng tham chiếu trực tiếp đến đối tượng, đồng thời đánh giá các biện pháp kiểm soát truy cập hiện có và xác định khả năng tồn tại lỗ hổng IDOR.
\\ \hline

WSTG-ATHZ-05 &
Kiểm thử các điểm yếu liên quan đến OAuth &
Đánh giá việc triển khai OAuth2 của ứng dụng nhằm xác định các điểm yếu bảo mật, bao gồm việc sử dụng các cơ chế đã lỗi thời, triển khai tùy chỉnh không an toàn hoặc cấu hình sai.
\\ \hline
\end{longtable}

\section{Kiểm thử quản lý phiên làm việc}
\begin{longtable}{|>{\centering\arraybackslash}p{1.3cm}|p{4.5cm}|p{7.7cm}|}

\caption{Mục đích của nội dung kiểm thử quản lý phiên làm việc}\\
\hline
\textbf{Mã} & \textbf{Nội dung kiểm thử} & \textbf{Mục tiêu} \\
\hline
\endfirsthead

\hline
\textbf{Mã} & \textbf{Nội dung kiểm thử} & \textbf{Mục tiêu} \\
\hline
\endhead

WSTG-SESS-01 &
Kiểm thử cơ chế quản lý phiên làm việc &
Thu thập các session token cho cùng một người dùng và giữa các người dùng khác nhau (nếu có thể), từ đó phân tích mức độ ngẫu nhiên của token nhằm ngăn chặn các tấn công giả mạo phiên. Đồng thời thử thay đổi các cookie không được ký số và chứa dữ liệu có thể bị thao túng để đánh giá mức độ an toàn.
\\ \hline

WSTG-SESS-02 &
Kiểm thử thuộc tính bảo mật của cookie &
Xác minh rằng các thuộc tính bảo mật cần thiết của cookie được cấu hình đúng, bao gồm nhưng không giới hạn ở các thuộc tính như HttpOnly, Secure và SameSite.
\\ \hline

WSTG-SESS-03 &
Kiểm thử lỗ hổng cố định phiên (Session Fixation) &
Phân tích cơ chế và luồng xác thực của ứng dụng, đồng thời thử ép sử dụng cookie phiên xác định trước để đánh giá mức độ ảnh hưởng và khả năng khai thác.
\\ \hline

WSTG-SESS-04 &
Kiểm thử việc lộ biến phiên &
Xác minh rằng các cơ chế mã hóa được triển khai đầy đủ, đồng thời rà soát cấu hình bộ nhớ đệm và đánh giá mức độ an toàn của các kênh truyền và phương thức giao tiếp.
\\ \hline

WSTG-SESS-05 &
Kiểm thử giả mạo yêu cầu liên trang (CSRF) &
Xác định khả năng thực hiện các yêu cầu thay mặt người dùng mà không có sự chủ động của người dùng, từ đó đánh giá mức độ ảnh hưởng của lỗ hổng CSRF nếu tồn tại.
\\ \hline

WSTG-SESS-06 &
Kiểm thử chức năng đăng xuất &
Đánh giá giao diện và hành vi của chức năng đăng xuất, phân tích thời gian tồn tại của phiên và xác minh rằng phiên làm việc được hủy hoàn toàn sau khi người dùng đăng xuất.
\\ \hline

WSTG-SESS-07 &
Kiểm thử thời gian hết hạn phiên &
Xác minh sự tồn tại của cơ chế hết hạn phiên cứng (hard session timeout) nhằm đảm bảo phiên làm việc không tồn tại vô thời hạn.
\\ \hline

WSTG-SESS-08 &
Kiểm thử làm rối luồng phiên &
Xác định tất cả các biến liên quan đến phiên và thử phá vỡ luồng logic của quá trình sinh phiên nhằm đánh giá khả năng dự đoán hoặc thao túng phiên.
\\ \hline

WSTG-SESS-09 &
Kiểm thử chiếm quyền phiên &
Xác định các cookie phiên dễ bị tấn công, thử chiếm quyền các cookie này và đánh giá mức độ rủi ro đối với hệ thống.
\\ \hline

WSTG-SESS-10 &
Kiểm thử JSON Web Token (JWT) &
Xác định liệu các JWT có làm lộ thông tin nhạy cảm hay không, đồng thời đánh giá khả năng bị chỉnh sửa hoặc giả mạo của JWT.
\\ \hline

WSTG-SESS-11 &
Kiểm thử phiên làm việc đồng thời &
Đánh giá cơ chế quản lý phiên của ứng dụng thông qua việc xử lý nhiều phiên hoạt động đồng thời cho cùng một tài khoản người dùng.
\\ \hline
\end{longtable}

\section{Kiểm thử xác thực dữ liệu đầu vào}
\begin{longtable}{|>{\centering\arraybackslash}p{1.3cm}|p{4.5cm}|p{7.7cm}|}

\caption{Mục đích của nội dung kiểm thử xác thực dữ liệu đầu vào}\\
\hline
\textbf{Mã} & \textbf{Nội dung kiểm thử} & \textbf{Mục tiêu} \\
\hline
\endfirsthead

\hline
\textbf{Mã} & \textbf{Nội dung kiểm thử} & \textbf{Mục tiêu} \\
\hline
\endhead

WSTG-INPV-01 &
Kiểm thử Cross-Site Scripting phản xạ (Reflected XSS) &
Xác định các biến đầu vào được phản xạ trong phản hồi của ứng dụng, đồng thời đánh giá loại dữ liệu đầu vào được chấp nhận và cơ chế mã hóa được áp dụng khi dữ liệu được trả về cho phía client.
\\ \hline

WSTG-INPV-02 &
Kiểm thử Cross-Site Scripting lưu trữ (Stored XSS) &
Xác định các dữ liệu đầu vào được lưu trữ và hiển thị lại phía client, đồng thời đánh giá cơ chế xử lý và mã hóa dữ liệu đầu vào khi được hiển thị.
\\ \hline

WSTG-INPV-03 &
Kiểm thử thay đổi HTTP Verb &
Đánh giá khả năng thay đổi phương thức HTTP nhằm vượt qua các cơ chế kiểm soát đầu vào hoặc logic xử lý phía server.
\\ \hline

WSTG-INPV-04 &
Kiểm thử ô nhiễm tham số HTTP (HTTP Parameter Pollution) &
Xác định backend và cơ chế phân tích tham số được sử dụng, đồng thời đánh giá các điểm có khả năng xảy ra chèn tham số và thử vượt qua bộ lọc đầu vào thông qua kỹ thuật HPP.
\\ \hline

WSTG-INPV-05 &
Kiểm thử SQL Injection &
Xác định các điểm có khả năng xảy ra SQL Injection, đồng thời đánh giá mức độ nghiêm trọng của lỗ hổng và phạm vi truy cập có thể đạt được thông qua khai thác.
\\ \hline

WSTG-INPV-06 &
Kiểm thử LDAP Injection &
Xác định các điểm có khả năng xảy ra LDAP Injection và đánh giá mức độ ảnh hưởng của lỗ hổng nếu tồn tại.
\\ \hline

WSTG-INPV-07 &
Kiểm thử XML Injection &
Xác định các điểm có khả năng xảy ra XML Injection, đồng thời đánh giá các hình thức khai thác có thể thực hiện và mức độ nghiêm trọng tương ứng.
\\ \hline

WSTG-INPV-08 &
Kiểm thử SSI Injection &
Xác định các điểm có khả năng xảy ra SSI Injection và đánh giá mức độ ảnh hưởng của lỗ hổng.
\\ \hline

WSTG-INPV-09 &
Kiểm thử XPath Injection &
Xác định các điểm có khả năng xảy ra XPath Injection trong quá trình xử lý dữ liệu XML của ứng dụng.
\\ \hline

WSTG-INPV-10 &
Kiểm thử IMAP/SMTP Injection &
Xác định các điểm có khả năng xảy ra IMAP/SMTP Injection, đồng thời phân tích luồng dữ liệu và kiến trúc triển khai của hệ thống để đánh giá mức độ ảnh hưởng.
\\ \hline

WSTG-INPV-11 &
Kiểm thử Code Injection &
Xác định các điểm có khả năng chèn mã vào ứng dụng và đánh giá mức độ nghiêm trọng của lỗ hổng nếu bị khai thác.
\\ \hline

WSTG-INPV-12 &
Kiểm thử Command Injection &
Xác định và đánh giá các điểm có khả năng xảy ra Command Injection trong ứng dụng.
\\ \hline


WSTG-INPV-13 &
Kiểm thử Format String Injection &
Đánh giá liệu việc chèn các định dạng chuỗi vào các trường dữ liệu do người dùng kiểm soát có gây ra hành vi không mong muốn từ ứng dụng hay không.
\\ \hline

WSTG-INPV-14 &
Kiểm thử lỗ hổng Incubated Vulnerability &
Xác định các điểm chèn dữ liệu được lưu trữ và chỉ được kích hoạt ở bước xử lý sau, đồng thời phân tích cơ chế kích hoạt và khả năng khai thác của lỗ hổng.
\\ \hline

WSTG-INPV-15 &
Kiểm thử HTTP Splitting và Smuggling &
Đánh giá khả năng ứng dụng bị tấn công HTTP Splitting hoặc Smuggling, đồng thời phân tích chuỗi giao tiếp để xác định các hình thức tấn công có thể xảy ra.
\\ \hline

WSTG-INPV-16 &
Giám sát các HTTP request đến &
Theo dõi toàn bộ các HTTP request vào và ra khỏi Web Server nhằm phát hiện các yêu cầu bất thường hoặc có dấu hiệu tấn công.
\\ \hline

WSTG-INPV-17 &
Kiểm thử Host Header Injection &
Đánh giá việc tiêu đề Host có được xử lý động trong ứng dụng hay không, đồng thời thử vượt qua các cơ chế bảo mật phụ thuộc vào tiêu đề này.
\\ \hline

WSTG-INPV-18 &
Kiểm thử Server-Side Template Injection &
Xác định các điểm có khả năng xảy ra Server-Side Template Injection, nhận diện engine template được sử dụng và đánh giá khả năng xây dựng khai thác.
\\ \hline

WSTG-INPV-19 &
Kiểm thử Server-Side Request Forgery &
Xác định các điểm có khả năng xảy ra SSRF, đánh giá khả năng khai thác và mức độ nghiêm trọng của lỗ hổng.
\\ \hline

WSTG-INPV-20 &
Kiểm thử Mass Assignment &
Xác định các request có khả năng thay đổi đối tượng và đánh giá việc sửa đổi các trường dữ liệu không được phép từ phía bên ngoài.
\\ \hline

\end{longtable}

\section{Kiểm thử xử lý lỗi}
\begin{longtable}{|>{\centering\arraybackslash}p{1.3cm}|p{4.5cm}|p{7.7cm}|}

\caption{Mục đích của nội dung kiểm thử xử lý lỗi}\\
\hline
\textbf{Mã} & \textbf{Nội dung kiểm thử} & \textbf{Mục tiêu} \\
\hline
\endfirsthead

\hline
\textbf{Mã} & \textbf{Nội dung kiểm thử} & \textbf{Mục tiêu} \\
\hline
\endhead

WSTG-ERRH-01 &
Kiểm thử xử lý lỗi không đúng &
Xác định các thông báo lỗi đang được ứng dụng trả về, đồng thời phân tích sự khác biệt giữa các phản hồi lỗi nhằm đánh giá nguy cơ rò rỉ thông tin hoặc hành vi xử lý lỗi không an toàn.
\\ \hline

WSTG-ERRH-02 &
Kiểm thử lộ stack trace &
Xác định việc ứng dụng có trả về stack trace hoặc thông tin chi tiết về lỗi trong phản hồi hay không, từ đó đánh giá khả năng lộ thông tin nội bộ và tác động bảo mật tương ứng.
\\ \hline

\end{longtable}

\section{Kiểm thử mật mã yếu}
\begin{longtable}{|>{\centering\arraybackslash}p{1.3cm}|p{4.5cm}|p{7.7cm}|}

\caption{Mục đích của nội dung kiểm thử mật mã yếu}\\
\hline
\textbf{Mã} & \textbf{Nội dung kiểm thử} & \textbf{Mục tiêu} \\
\hline
\endfirsthead

\hline
\textbf{Mã} & \textbf{Nội dung kiểm thử} & \textbf{Mục tiêu} \\
\hline
\endhead

WSTG-CRYP-01 &
Kiểm thử bảo mật tầng truyền tải yếu &
Xác minh cấu hình dịch vụ liên quan đến bảo mật tầng truyền tải, đánh giá độ mạnh và tính hợp lệ của chứng chỉ số, đồng thời đảm bảo cơ chế bảo mật TLS được triển khai đúng cách và không thể bị vượt qua trong toàn bộ ứng dụng.
\\ \hline

WSTG-CRYP-02 &
Kiểm thử Padding Oracle &
Xác định các thông điệp đã được mã hóa có sử dụng cơ chế padding, đồng thời thử phá vỡ padding của các thông điệp này và phân tích các thông báo lỗi được trả về nhằm phục vụ cho việc đánh giá mức độ an toàn của cơ chế mã hóa.
\\ \hline

WSTG-CRYP-03 &
Kiểm thử truyền thông tin nhạy cảm qua kênh không mã hóa &
Xác định các thông tin nhạy cảm được truyền qua các kênh không được mã hóa và đánh giá mức độ riêng tư cũng như mức độ an toàn của các kênh truyền thông đang được sử dụng.
\\ \hline

WSTG-CRYP-04 &
Kiểm thử cơ chế mã hóa yếu &
Đánh giá việc sử dụng các thuật toán mã hóa hoặc hàm băm yếu, cũng như các cách triển khai không an toàn có thể làm suy giảm mức độ bảo mật của hệ thống.
\\ \hline

\end{longtable}

\section{Kiểm thử phía máy khách}
\begin{longtable}{|>{\centering\arraybackslash}p{1.3cm}|p{4.5cm}|p{7.7cm}|}

\caption{Mục đích của nội dung kiểm thử phía máy khách}\\
\hline
\textbf{Mã} & \textbf{Nội dung kiểm thử} & \textbf{Mục tiêu} \\
\hline
\endfirsthead

\hline
\textbf{Mã} & \textbf{Nội dung kiểm thử} & \textbf{Mục tiêu} \\
\hline
\endhead

WSTG-CLNT-01 &
Kiểm thử DOM-Based Cross-Site Scripting &
Xác định các DOM sink tồn tại trong ứng dụng và xây dựng các payload phù hợp với từng loại sink nhằm đánh giá khả năng khai thác lỗ hổng DOM-based XSS.
\\ \hline

WSTG-CLNT-02 &
Kiểm thử thực thi JavaScript &
Xác định các sink và các điểm có khả năng xảy ra JavaScript Injection trong quá trình thực thi mã phía client.
\\ \hline

WSTG-CLNT-03 &
Kiểm thử HTML Injection &
Xác định các điểm có khả năng xảy ra HTML Injection và đánh giá mức độ ảnh hưởng của nội dung được chèn vào ứng dụng.
\\ \hline

WSTG-CLNT-04 &
Kiểm thử chuyển hướng URL phía client &
Xác định các điểm xử lý URL hoặc đường dẫn trên phía client và đánh giá các vị trí mà hệ thống có thể bị chuyển hướng tới.
\\ \hline

WSTG-CLNT-05 &
Kiểm thử CSS Injection &
Xác định các điểm có khả năng xảy ra CSS Injection và đánh giá mức độ ảnh hưởng của việc chèn CSS đối với ứng dụng.
\\ \hline

WSTG-CLNT-06 &
Kiểm thử thao túng tài nguyên phía client &
Xác định các sink có cơ chế kiểm tra dữ liệu đầu vào yếu, đồng thời đánh giá tác động của việc thao túng tài nguyên phía client.
\\ \hline

WSTG-CLNT-07 &
Kiểm thử chia sẻ tài nguyên khác nguồn (CORS) &
Xác định các endpoint triển khai cơ chế CORS và đánh giá việc cấu hình CORS có an toàn hoặc không gây ảnh hưởng tiêu cực đến bảo mật hệ thống.
\\ \hline

WSTG-CLNT-08 &
Kiểm thử Cross-Site Flashing &
Thực hiện phân tích và giải mã mã nguồn của ứng dụng (nếu có), đồng thời đánh giá dữ liệu đầu vào tại các sink và việc sử dụng các phương thức không an toàn.
\\ \hline

WSTG-CLNT-09 &
Kiểm thử Clickjacking &
Đánh giá mức độ dễ bị tấn công clickjacking của ứng dụng thông qua các cơ chế hiển thị và nhúng nội dung.
\\ \hline

WSTG-CLNT-10 &
Kiểm thử WebSockets &
Xác định việc sử dụng WebSocket trong ứng dụng và đánh giá cách thức triển khai bằng cách áp dụng các bài kiểm thử tương tự như trên các kênh HTTP thông thường.
\\ \hline

WSTG-CLNT-11 &
Kiểm thử Web Messaging &
Đánh giá mức độ an toàn của nguồn gốc thông điệp, đồng thời xác minh rằng ứng dụng sử dụng các phương thức an toàn và có cơ chế kiểm tra dữ liệu đầu vào phù hợp.
\\ \hline

WSTG-CLNT-12 &
Kiểm thử lưu trữ trình duyệt &
Xác định liệu website có lưu trữ dữ liệu nhạy cảm trong các cơ chế lưu trữ phía client hay không, đồng thời phân tích cách xử lý các đối tượng lưu trữ để phát hiện khả năng chèn mã.
\\ \hline

WSTG-CLNT-13 &
Kiểm thử Cross-Site Script Inclusion &
Xác định các dữ liệu nhạy cảm tồn tại trong hệ thống và đánh giá nguy cơ rò rỉ thông tin thông qua các kỹ thuật khai thác khác nhau.
\\ \hline

WSTG-CLNT-14 &
Kiểm thử Reverse Tabnabbing &
Đánh giá khả năng ứng dụng bị tấn công reverse tabnabbing thông qua các liên kết mở tab mới không được kiểm soát.
\\ \hline

\end{longtable}


\end{document}