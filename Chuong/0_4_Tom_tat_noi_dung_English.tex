\documentclass[../DoAn.tex]{subfiles}
\begin{document}

\begin{center}
    \Large{\textbf{ABSTRACT}}\\
\end{center}
\vspace{1cm}

In the context of rapid digital transformation today, ensuring the security of web applications plays a crucial role in protecting user data and systems from increasingly sophisticated cyber threats. To reduce risks and minimize security-related incidents, the security testing process must be integrated as an essential part of the software development lifecycle.

This thesis focuses on researching and implementing web security testing — a key area in cybersecurity — to evaluate the security level of applications and identify vulnerabilities that could be exploited. Throughout its operation, any system may develop weaknesses, and if not detected in time, these vulnerabilities can lead to serious consequences. Therefore, conducting periodic penetration testing is necessary to ensure the security of applications throughout their lifecycle.

The education sector, which stores a large amount of sensitive data, is always an attractive target for cybercriminals. The Learning Management System (LMS) of Hanoi University of Science and Technology (lms.hust.edu.vn) is one of the most essential platforms used daily by lecturers and students; hence, a comprehensive security assessment is required. This thesis applies the OWASP (Open Web Application Security Project) testing framework to evaluate and perform security testing on the LMS system. The OWASP standards provide a scientific, structured, and modern methodology that ensures the testing process is systematic and effective.

This thesis consists of 6 main chapters:
\begin{itemize}
\item \textbf{Chapter 1: Introduction}
\item \textbf{Chapter 2: Testing Methodology}
\item \textbf{Chapter 3: Testing Procedure}
\item \textbf{Chapter 4: Solutions and Contributions}
\item \textbf{Chapter 5: Conclusion}
\end{itemize}



\end{document}